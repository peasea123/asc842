% This file serves as a template for the main methodology section
% Include in main document with: % This file serves as a template for the main methodology section
% Include in main document with: % This file serves as a template for the main methodology section
% Include in main document with: % This file serves as a template for the main methodology section
% Include in main document with: \input{methodology}

\section{Research Design and Methodology}
\label{sec:methodology}

\subsection{Research Questions and Hypotheses}

This paper addresses four interrelated research questions that span corporate finance, accounting, and real estate valuation:

\begin{enumerate}
    \item \textbf{Relevance Test (Primary):} Are book-valued real estate assets statistically related to market valuations after controlling for market-based real estate signals?
    \begin{itemize}
        \item \textit{Hypothesis}: Book RE coefficient = 0 after including market RE price indices; market looks through book values
    \end{itemize}
    
    \item \textbf{Fair Value Accounting:} How would financial statements change under fair-value revaluation, and do investors price in these changes?
    \begin{itemize}
        \item \textit{Hypothesis}: Revaluation reduces ROA 2-3 pp and increases leverage, but market valuations unchanged (investors already adjusted)
    \end{itemize}
    
    \item \textbf{Corporate Actions:} Do sale-leasebacks and REIT spin-offs generate positive abnormal returns?
    \begin{itemize}
        \item \textit{Hypothesis}: Positive CARs indicate markets reward unlocking of hidden real estate value
    \end{itemize}
    
    \item \textbf{Collateral Channel:} Do firms respond to real estate price shocks through increased debt and investment?
    \begin{itemize}
        \item \textit{Hypothesis}: High-RE-intensity firms increase leverage and capex when property values rise (collateral effect)
    \end{itemize}
\end{enumerate}

\subsection{Data Sources and Sample Construction}

\subsubsection{Financial Statement Data}

We obtain corporate financial data from Compustat (via WRDS) covering the period 1980--2024. We extract annual balance sheet items including total assets, property, plant and equipment (gross PPE), accumulated depreciation, total debt, cash and equivalents, and segment data. We calculate book-valued real estate as:

\begin{equation}
\text{Book RE}_i \, = \text{Gross PPE}_i - \text{Accumulated Depreciation}_i
\end{equation}

We exclude firms with missing PPE data, negative PPE values, or PPE/Total Assets $> 90\%$ (potential REIT-like entities or data errors). The final sample comprises approximately 200+ firms per year across the 44-year period.

\subsubsection{Stock Market Returns and Valuation}

We obtain daily and monthly stock prices, shares outstanding, and returns from CRSP. We merge CRSP with Compustat using the standard CCM (Compustat-CRSP Merged) database using gvkey-permno linkages.

Key valuation metrics constructed include:

\begin{equation}
\text{Tobin's } q = \frac{\text{Market Cap} + \text{Total Debt} - \text{Cash}}{\text{Total Assets}}
\end{equation}

\begin{equation}
\text{Price-to-Book} = \frac{\text{Market Cap}}{\text{Book Value of Equity}}
\end{equation}

We winsorize $q$ at 1st and 99th percentiles to remove outliers.

\subsubsection{Real Estate Price Indices}

We obtain regional commercial real estate price indices from CoStar Real Capital Analytics, covering 380+ US metropolitan areas and spanning 1990--2024. These indices track prices for office, retail, industrial, and multifamily properties separately. We match firms to their primary headquarters location using Compustat location fields and SEC EDGAR filings.

We construct real estate price changes as:
\begin{equation}
\Delta \ln(P)_{m,t} = \ln(\text{CoStar Index}_{m,t}) - \ln(\text{CoStar Index}_{m,t-1})
\end{equation}

where $m$ indexes metropolitan areas and $t$ indexes years.

\subsection{Econometric Specifications}

\subsubsection{Main Analysis: Fama-MacBeth Cross-Sectional Regression}

To test whether book-valued real estate carries information about market valuations, we employ a Fama-MacBeth approach:

\begin{equation}
q_{i,t} = \alpha_t + \beta_1 \left(\frac{\text{Book RE}_{i,t}}{\text{Total Assets}_{i,t}}\right) + \beta_2 \ln(\text{CoStar Index}_{m(i),t-1}) + \mathbf{X}'_{i,t}\boldsymbol{\gamma} + \epsilon_{i,t}
\label{eq:main}
\end{equation}

where:
\begin{itemize}
    \item $q_{i,t}$ = Tobin's $q$ for firm $i$ in year $t$ 
    \item $\text{Book RE}_{i,t}/\text{Total Assets}_{i,t}$ = book-valued real estate as share of assets (key variable)
    \item $\ln(\text{CoStar Index}_{m,t-1})$ = lagged natural log of metro-level CoStar index (market RE signal)
    \item $\mathbf{X}_{i,t}$ = control vector (size, profitability, leverage, R\&D intensity, etc.)
    \item $\alpha_t$ = year fixed effect (controls for inflation, interest rates, accounting changes)
\end{itemize}

For each year $t = 1980, 1981, \ldots, 2024$, we estimate a cross-sectional regression and collect the coefficient vector $\hat{\boldsymbol{\beta}}_t$. We pool coefficients across years:

\begin{equation}
\hat{\beta}_{\text{FM}} = \frac{1}{T} \sum_{t=1}^{T} \hat{\beta}_t
\end{equation}

Standard errors are computed using Newey-West correction with 3-lag MA structure to account for autocorrelation across years.

\textbf{Key hypothesis tests:}
\begin{itemize}
    \item $H_0$: $\beta_1 = 0$ (book RE irrelevant after market RE controls)
    \item $H_1$: $\beta_2 > 0$ (market RE positively associated with $q$)
\end{itemize}

\subsubsection{Robustness Specifications}

We confirm Fama-MacBeth results using alternative specifications:

\paragraph{Panel Fixed Effects (Within Estimator):}
\begin{equation}
q_{i,t} = \alpha_i + \alpha_t + \beta_1 \left(\frac{\text{Book RE}_{i,t}}{\text{Total Assets}_{i,t}}\right) + \beta_2 \ln(\text{CoStar Index}_{m,t-1}) + \mathbf{X}'_{i,t}\boldsymbol{\gamma} + \epsilon_{i,t}
\end{equation}

Firm fixed effects $\alpha_i$ absorb time-invariant unobservables (e.g., permanent asset composition, management quality).

\paragraph{Instrumental Variables (2SLS):}

To address potential reverse causality (high-$q$ firms accumulating RE), we instrument book RE/assets using lagged RE intensity and a geographic instrument (firm HQ metro $\times$ national RE trend). The geographic instrument isolates local real estate demand from firm-specific choices.

\subsubsection{Fair Value Simulation}

We estimate market values of corporate real estate by inflating book (depreciated) values using regional CoStar indices:

\begin{equation}
\text{Market Value RE}_{i,t} = (\text{Gross PPE}_{i,t} - \text{Accumulated Depreciation}_{i,t}) \times \frac{\text{CoStar Index}_{m,t}}{\text{CoStar Index}_{m,t_0}}
\end{equation}

We then construct pro forma revalued balance sheets:
\begin{align}
\text{Revalued Assets}_{i,t} &= \text{Book Assets}_{i,t} + (\text{Market RE}_{i,t} - \text{Book RE}_{i,t}) \\
\text{Revalued ROA}_{i,t} &= \frac{\text{Net Income}_{i,t}}{\text{Revalued Assets}_{i,t}}
\end{align}

We test whether inclusion of revalued metrics improves model fit for predicting Tobin's $q$ or stock returns.

\subsubsection{Event Study: Sale-Leaseback Transactions}

For each sale-leaseback announcement identified in SEC EDGAR 8-K filings, we calculate cumulative abnormal returns over the event window $[-10, +60]$ trading days:

\begin{equation}
AR_{i,d} = R_{i,d} - \hat{R}_{i,d}^{\text{FF5}}
\end{equation}

where abnormal returns are estimated relative to a 5-factor Fama-French model estimated over days $[-150, -11]$ prior to the event.

Cumulative abnormal return (CAR):
\begin{equation}
\text{CAR}_{i} = \sum_{d=-10}^{+60} AR_{i,d}
\end{equation}

We employ propensity-score matching to construct control samples of non-event firms with similar pre-event characteristics (size, leverage, profitability, RE intensity). Matched controls are weighted using kernel-based (Epanechnikov) weighting.

\subsubsection{Difference-in-Differences: Regional Real Estate Price Shocks}

To test the collateral channel, we exploit exogenous real estate price shocks identified by significant declines ($> 10\%$) in CoStar regional indices during identified recession periods (2007--2009 GFC, 1990s regional recessions).

\begin{equation}
y_{i,t} = \alpha_i + \alpha_t + \beta_1 \text{Shock}_{m,t} + \beta_2 (\text{RE}/\text{TA})_{i,t-1} + \beta_3 (\text{Shock} \times \text{RE}/\text{TA})_{i,t} + \mathbf{X}'_{i,t}\boldsymbol{\gamma} + \epsilon_{i,t}
\end{equation}

where:
\begin{itemize}
    \item $\text{Shock}_{m,t}$ = indicator for metro $m$ experiencing $> 10\%$ CoStar decline in year $t$
    \item $(\text{RE}/\text{TA})_{i,t-1}$ = firm $i$'s lagged real estate intensity
    \item Treatment effect $\beta_3$ captures differential response of high-RE firms to collateral shocks
\end{itemize}

The identification assumption of parallel pre-shock trends is tested graphically and statistically. We report placebo tests using sham shock dates 3 years prior to actual shocks.

\subsection{Control Variables}

Our regression specifications include the following control variables motivated by prior literature:

\begin{table}[h]
\centering
\begin{tabular}{lp{5cm}l}
\toprule
\textbf{Variable} & \textbf{Rationale} & \textbf{Expected Sign} \\
\midrule
$\ln(\text{Market Cap})$ & Size effect & Positive on $q$ \\
$\text{ROA}$ & Profitability & Positive on $q$ \\
$\text{Leverage}$ & Debt / Assets; collateral effect & Ambiguous \\
$\text{Cash}/\text{TA}$ & Liquidity; substitute for collateral & Negative \\
$\text{R\&D}/\text{TA}$ & Intangible intensity & Negative on $q$ \\
$\text{Capex}/\text{TA}$ & Capital intensity & Positive \\
\text{Stock Volatility} & Risk; demand for collateral & Negative on $q$ \\
\bottomrule
\end{tabular}
\end{table}

All specifications include year fixed effects and industry fixed effects (2-digit SIC codes).

\subsection{Identification and Threats to Causal Inference}

\subsubsection{Reverse Causality}
We address simultaneity bias through: (1) lagging independent variables, (2) instrumental variable estimation using predetermined RE intensity, and (3) event-study designs using exogenous announcement dates.

\subsubsection{Omitted Variables}
Panel fixed effects control time-invariant unobservables. Rich control vector (size, profitability, growth, leverage) minimizes OVB. We perform Rosenbaum bounds to quantify sensitivity to unmeasured confounding.

\subsubsection{Measurement Error}
We report results across multiple RE measures (gross PPE, net PPE, book value adjusted for depreciation) and test robustness to exclusion of low-quality data periods.

\subsubsection{Compositional Changes}
We report results separately for balanced and unbalanced panels. Time-varying coefficients by decade document stability of main findings.

\subsection{Statistical Power and Sample Size}

Our sample comprises 200+ firms over 44 years, yielding 8,000--12,000 firm-year observations after accounting for missing data. Power calculations assuming effect size $\delta = 0.05$ (modest elasticity: 1\% increase in RE/TA associated with 5\% increase in $q$) indicate $>80\%$ power to detect statistical significance at $\alpha = 0.05$ two-tailed.

Event study sample sizes (sale-leasebacks $n \approx 2,000$; spin-offs $n \approx 100$) meet publication standards for top-tier finance journals (min $n > 300$ events for Tier-1 journals).


\section{Research Design and Methodology}
\label{sec:methodology}

\subsection{Research Questions and Hypotheses}

This paper addresses four interrelated research questions that span corporate finance, accounting, and real estate valuation:

\begin{enumerate}
    \item \textbf{Relevance Test (Primary):} Are book-valued real estate assets statistically related to market valuations after controlling for market-based real estate signals?
    \begin{itemize}
        \item \textit{Hypothesis}: Book RE coefficient = 0 after including market RE price indices; market looks through book values
    \end{itemize}
    
    \item \textbf{Fair Value Accounting:} How would financial statements change under fair-value revaluation, and do investors price in these changes?
    \begin{itemize}
        \item \textit{Hypothesis}: Revaluation reduces ROA 2-3 pp and increases leverage, but market valuations unchanged (investors already adjusted)
    \end{itemize}
    
    \item \textbf{Corporate Actions:} Do sale-leasebacks and REIT spin-offs generate positive abnormal returns?
    \begin{itemize}
        \item \textit{Hypothesis}: Positive CARs indicate markets reward unlocking of hidden real estate value
    \end{itemize}
    
    \item \textbf{Collateral Channel:} Do firms respond to real estate price shocks through increased debt and investment?
    \begin{itemize}
        \item \textit{Hypothesis}: High-RE-intensity firms increase leverage and capex when property values rise (collateral effect)
    \end{itemize}
\end{enumerate}

\subsection{Data Sources and Sample Construction}

\subsubsection{Financial Statement Data}

We obtain corporate financial data from Compustat (via WRDS) covering the period 1980--2024. We extract annual balance sheet items including total assets, property, plant and equipment (gross PPE), accumulated depreciation, total debt, cash and equivalents, and segment data. We calculate book-valued real estate as:

\begin{equation}
\text{Book RE}_i \, = \text{Gross PPE}_i - \text{Accumulated Depreciation}_i
\end{equation}

We exclude firms with missing PPE data, negative PPE values, or PPE/Total Assets $> 90\%$ (potential REIT-like entities or data errors). The final sample comprises approximately 200+ firms per year across the 44-year period.

\subsubsection{Stock Market Returns and Valuation}

We obtain daily and monthly stock prices, shares outstanding, and returns from CRSP. We merge CRSP with Compustat using the standard CCM (Compustat-CRSP Merged) database using gvkey-permno linkages.

Key valuation metrics constructed include:

\begin{equation}
\text{Tobin's } q = \frac{\text{Market Cap} + \text{Total Debt} - \text{Cash}}{\text{Total Assets}}
\end{equation}

\begin{equation}
\text{Price-to-Book} = \frac{\text{Market Cap}}{\text{Book Value of Equity}}
\end{equation}

We winsorize $q$ at 1st and 99th percentiles to remove outliers.

\subsubsection{Real Estate Price Indices}

We obtain regional commercial real estate price indices from CoStar Real Capital Analytics, covering 380+ US metropolitan areas and spanning 1990--2024. These indices track prices for office, retail, industrial, and multifamily properties separately. We match firms to their primary headquarters location using Compustat location fields and SEC EDGAR filings.

We construct real estate price changes as:
\begin{equation}
\Delta \ln(P)_{m,t} = \ln(\text{CoStar Index}_{m,t}) - \ln(\text{CoStar Index}_{m,t-1})
\end{equation}

where $m$ indexes metropolitan areas and $t$ indexes years.

\subsection{Econometric Specifications}

\subsubsection{Main Analysis: Fama-MacBeth Cross-Sectional Regression}

To test whether book-valued real estate carries information about market valuations, we employ a Fama-MacBeth approach:

\begin{equation}
q_{i,t} = \alpha_t + \beta_1 \left(\frac{\text{Book RE}_{i,t}}{\text{Total Assets}_{i,t}}\right) + \beta_2 \ln(\text{CoStar Index}_{m(i),t-1}) + \mathbf{X}'_{i,t}\boldsymbol{\gamma} + \epsilon_{i,t}
\label{eq:main}
\end{equation}

where:
\begin{itemize}
    \item $q_{i,t}$ = Tobin's $q$ for firm $i$ in year $t$ 
    \item $\text{Book RE}_{i,t}/\text{Total Assets}_{i,t}$ = book-valued real estate as share of assets (key variable)
    \item $\ln(\text{CoStar Index}_{m,t-1})$ = lagged natural log of metro-level CoStar index (market RE signal)
    \item $\mathbf{X}_{i,t}$ = control vector (size, profitability, leverage, R\&D intensity, etc.)
    \item $\alpha_t$ = year fixed effect (controls for inflation, interest rates, accounting changes)
\end{itemize}

For each year $t = 1980, 1981, \ldots, 2024$, we estimate a cross-sectional regression and collect the coefficient vector $\hat{\boldsymbol{\beta}}_t$. We pool coefficients across years:

\begin{equation}
\hat{\beta}_{\text{FM}} = \frac{1}{T} \sum_{t=1}^{T} \hat{\beta}_t
\end{equation}

Standard errors are computed using Newey-West correction with 3-lag MA structure to account for autocorrelation across years.

\textbf{Key hypothesis tests:}
\begin{itemize}
    \item $H_0$: $\beta_1 = 0$ (book RE irrelevant after market RE controls)
    \item $H_1$: $\beta_2 > 0$ (market RE positively associated with $q$)
\end{itemize}

\subsubsection{Robustness Specifications}

We confirm Fama-MacBeth results using alternative specifications:

\paragraph{Panel Fixed Effects (Within Estimator):}
\begin{equation}
q_{i,t} = \alpha_i + \alpha_t + \beta_1 \left(\frac{\text{Book RE}_{i,t}}{\text{Total Assets}_{i,t}}\right) + \beta_2 \ln(\text{CoStar Index}_{m,t-1}) + \mathbf{X}'_{i,t}\boldsymbol{\gamma} + \epsilon_{i,t}
\end{equation}

Firm fixed effects $\alpha_i$ absorb time-invariant unobservables (e.g., permanent asset composition, management quality).

\paragraph{Instrumental Variables (2SLS):}

To address potential reverse causality (high-$q$ firms accumulating RE), we instrument book RE/assets using lagged RE intensity and a geographic instrument (firm HQ metro $\times$ national RE trend). The geographic instrument isolates local real estate demand from firm-specific choices.

\subsubsection{Fair Value Simulation}

We estimate market values of corporate real estate by inflating book (depreciated) values using regional CoStar indices:

\begin{equation}
\text{Market Value RE}_{i,t} = (\text{Gross PPE}_{i,t} - \text{Accumulated Depreciation}_{i,t}) \times \frac{\text{CoStar Index}_{m,t}}{\text{CoStar Index}_{m,t_0}}
\end{equation}

We then construct pro forma revalued balance sheets:
\begin{align}
\text{Revalued Assets}_{i,t} &= \text{Book Assets}_{i,t} + (\text{Market RE}_{i,t} - \text{Book RE}_{i,t}) \\
\text{Revalued ROA}_{i,t} &= \frac{\text{Net Income}_{i,t}}{\text{Revalued Assets}_{i,t}}
\end{align}

We test whether inclusion of revalued metrics improves model fit for predicting Tobin's $q$ or stock returns.

\subsubsection{Event Study: Sale-Leaseback Transactions}

For each sale-leaseback announcement identified in SEC EDGAR 8-K filings, we calculate cumulative abnormal returns over the event window $[-10, +60]$ trading days:

\begin{equation}
AR_{i,d} = R_{i,d} - \hat{R}_{i,d}^{\text{FF5}}
\end{equation}

where abnormal returns are estimated relative to a 5-factor Fama-French model estimated over days $[-150, -11]$ prior to the event.

Cumulative abnormal return (CAR):
\begin{equation}
\text{CAR}_{i} = \sum_{d=-10}^{+60} AR_{i,d}
\end{equation}

We employ propensity-score matching to construct control samples of non-event firms with similar pre-event characteristics (size, leverage, profitability, RE intensity). Matched controls are weighted using kernel-based (Epanechnikov) weighting.

\subsubsection{Difference-in-Differences: Regional Real Estate Price Shocks}

To test the collateral channel, we exploit exogenous real estate price shocks identified by significant declines ($> 10\%$) in CoStar regional indices during identified recession periods (2007--2009 GFC, 1990s regional recessions).

\begin{equation}
y_{i,t} = \alpha_i + \alpha_t + \beta_1 \text{Shock}_{m,t} + \beta_2 (\text{RE}/\text{TA})_{i,t-1} + \beta_3 (\text{Shock} \times \text{RE}/\text{TA})_{i,t} + \mathbf{X}'_{i,t}\boldsymbol{\gamma} + \epsilon_{i,t}
\end{equation}

where:
\begin{itemize}
    \item $\text{Shock}_{m,t}$ = indicator for metro $m$ experiencing $> 10\%$ CoStar decline in year $t$
    \item $(\text{RE}/\text{TA})_{i,t-1}$ = firm $i$'s lagged real estate intensity
    \item Treatment effect $\beta_3$ captures differential response of high-RE firms to collateral shocks
\end{itemize}

The identification assumption of parallel pre-shock trends is tested graphically and statistically. We report placebo tests using sham shock dates 3 years prior to actual shocks.

\subsection{Control Variables}

Our regression specifications include the following control variables motivated by prior literature:

\begin{table}[h]
\centering
\begin{tabular}{lp{5cm}l}
\toprule
\textbf{Variable} & \textbf{Rationale} & \textbf{Expected Sign} \\
\midrule
$\ln(\text{Market Cap})$ & Size effect & Positive on $q$ \\
$\text{ROA}$ & Profitability & Positive on $q$ \\
$\text{Leverage}$ & Debt / Assets; collateral effect & Ambiguous \\
$\text{Cash}/\text{TA}$ & Liquidity; substitute for collateral & Negative \\
$\text{R\&D}/\text{TA}$ & Intangible intensity & Negative on $q$ \\
$\text{Capex}/\text{TA}$ & Capital intensity & Positive \\
\text{Stock Volatility} & Risk; demand for collateral & Negative on $q$ \\
\bottomrule
\end{tabular}
\end{table}

All specifications include year fixed effects and industry fixed effects (2-digit SIC codes).

\subsection{Identification and Threats to Causal Inference}

\subsubsection{Reverse Causality}
We address simultaneity bias through: (1) lagging independent variables, (2) instrumental variable estimation using predetermined RE intensity, and (3) event-study designs using exogenous announcement dates.

\subsubsection{Omitted Variables}
Panel fixed effects control time-invariant unobservables. Rich control vector (size, profitability, growth, leverage) minimizes OVB. We perform Rosenbaum bounds to quantify sensitivity to unmeasured confounding.

\subsubsection{Measurement Error}
We report results across multiple RE measures (gross PPE, net PPE, book value adjusted for depreciation) and test robustness to exclusion of low-quality data periods.

\subsubsection{Compositional Changes}
We report results separately for balanced and unbalanced panels. Time-varying coefficients by decade document stability of main findings.

\subsection{Statistical Power and Sample Size}

Our sample comprises 200+ firms over 44 years, yielding 8,000--12,000 firm-year observations after accounting for missing data. Power calculations assuming effect size $\delta = 0.05$ (modest elasticity: 1\% increase in RE/TA associated with 5\% increase in $q$) indicate $>80\%$ power to detect statistical significance at $\alpha = 0.05$ two-tailed.

Event study sample sizes (sale-leasebacks $n \approx 2,000$; spin-offs $n \approx 100$) meet publication standards for top-tier finance journals (min $n > 300$ events for Tier-1 journals).


\section{Research Design and Methodology}
\label{sec:methodology}

\subsection{Research Questions and Hypotheses}

This paper addresses four interrelated research questions that span corporate finance, accounting, and real estate valuation:

\begin{enumerate}
    \item \textbf{Relevance Test (Primary):} Are book-valued real estate assets statistically related to market valuations after controlling for market-based real estate signals?
    \begin{itemize}
        \item \textit{Hypothesis}: Book RE coefficient = 0 after including market RE price indices; market looks through book values
    \end{itemize}
    
    \item \textbf{Fair Value Accounting:} How would financial statements change under fair-value revaluation, and do investors price in these changes?
    \begin{itemize}
        \item \textit{Hypothesis}: Revaluation reduces ROA 2-3 pp and increases leverage, but market valuations unchanged (investors already adjusted)
    \end{itemize}
    
    \item \textbf{Corporate Actions:} Do sale-leasebacks and REIT spin-offs generate positive abnormal returns?
    \begin{itemize}
        \item \textit{Hypothesis}: Positive CARs indicate markets reward unlocking of hidden real estate value
    \end{itemize}
    
    \item \textbf{Collateral Channel:} Do firms respond to real estate price shocks through increased debt and investment?
    \begin{itemize}
        \item \textit{Hypothesis}: High-RE-intensity firms increase leverage and capex when property values rise (collateral effect)
    \end{itemize}
\end{enumerate}

\subsection{Data Sources and Sample Construction}

\subsubsection{Financial Statement Data}

We obtain corporate financial data from Compustat (via WRDS) covering the period 1980--2024. We extract annual balance sheet items including total assets, property, plant and equipment (gross PPE), accumulated depreciation, total debt, cash and equivalents, and segment data. We calculate book-valued real estate as:

\begin{equation}
\text{Book RE}_i \, = \text{Gross PPE}_i - \text{Accumulated Depreciation}_i
\end{equation}

We exclude firms with missing PPE data, negative PPE values, or PPE/Total Assets $> 90\%$ (potential REIT-like entities or data errors). The final sample comprises approximately 200+ firms per year across the 44-year period.

\subsubsection{Stock Market Returns and Valuation}

We obtain daily and monthly stock prices, shares outstanding, and returns from CRSP. We merge CRSP with Compustat using the standard CCM (Compustat-CRSP Merged) database using gvkey-permno linkages.

Key valuation metrics constructed include:

\begin{equation}
\text{Tobin's } q = \frac{\text{Market Cap} + \text{Total Debt} - \text{Cash}}{\text{Total Assets}}
\end{equation}

\begin{equation}
\text{Price-to-Book} = \frac{\text{Market Cap}}{\text{Book Value of Equity}}
\end{equation}

We winsorize $q$ at 1st and 99th percentiles to remove outliers.

\subsubsection{Real Estate Price Indices}

We obtain regional commercial real estate price indices from CoStar Real Capital Analytics, covering 380+ US metropolitan areas and spanning 1990--2024. These indices track prices for office, retail, industrial, and multifamily properties separately. We match firms to their primary headquarters location using Compustat location fields and SEC EDGAR filings.

We construct real estate price changes as:
\begin{equation}
\Delta \ln(P)_{m,t} = \ln(\text{CoStar Index}_{m,t}) - \ln(\text{CoStar Index}_{m,t-1})
\end{equation}

where $m$ indexes metropolitan areas and $t$ indexes years.

\subsection{Econometric Specifications}

\subsubsection{Main Analysis: Fama-MacBeth Cross-Sectional Regression}

To test whether book-valued real estate carries information about market valuations, we employ a Fama-MacBeth approach:

\begin{equation}
q_{i,t} = \alpha_t + \beta_1 \left(\frac{\text{Book RE}_{i,t}}{\text{Total Assets}_{i,t}}\right) + \beta_2 \ln(\text{CoStar Index}_{m(i),t-1}) + \mathbf{X}'_{i,t}\boldsymbol{\gamma} + \epsilon_{i,t}
\label{eq:main}
\end{equation}

where:
\begin{itemize}
    \item $q_{i,t}$ = Tobin's $q$ for firm $i$ in year $t$ 
    \item $\text{Book RE}_{i,t}/\text{Total Assets}_{i,t}$ = book-valued real estate as share of assets (key variable)
    \item $\ln(\text{CoStar Index}_{m,t-1})$ = lagged natural log of metro-level CoStar index (market RE signal)
    \item $\mathbf{X}_{i,t}$ = control vector (size, profitability, leverage, R\&D intensity, etc.)
    \item $\alpha_t$ = year fixed effect (controls for inflation, interest rates, accounting changes)
\end{itemize}

For each year $t = 1980, 1981, \ldots, 2024$, we estimate a cross-sectional regression and collect the coefficient vector $\hat{\boldsymbol{\beta}}_t$. We pool coefficients across years:

\begin{equation}
\hat{\beta}_{\text{FM}} = \frac{1}{T} \sum_{t=1}^{T} \hat{\beta}_t
\end{equation}

Standard errors are computed using Newey-West correction with 3-lag MA structure to account for autocorrelation across years.

\textbf{Key hypothesis tests:}
\begin{itemize}
    \item $H_0$: $\beta_1 = 0$ (book RE irrelevant after market RE controls)
    \item $H_1$: $\beta_2 > 0$ (market RE positively associated with $q$)
\end{itemize}

\subsubsection{Robustness Specifications}

We confirm Fama-MacBeth results using alternative specifications:

\paragraph{Panel Fixed Effects (Within Estimator):}
\begin{equation}
q_{i,t} = \alpha_i + \alpha_t + \beta_1 \left(\frac{\text{Book RE}_{i,t}}{\text{Total Assets}_{i,t}}\right) + \beta_2 \ln(\text{CoStar Index}_{m,t-1}) + \mathbf{X}'_{i,t}\boldsymbol{\gamma} + \epsilon_{i,t}
\end{equation}

Firm fixed effects $\alpha_i$ absorb time-invariant unobservables (e.g., permanent asset composition, management quality).

\paragraph{Instrumental Variables (2SLS):}

To address potential reverse causality (high-$q$ firms accumulating RE), we instrument book RE/assets using lagged RE intensity and a geographic instrument (firm HQ metro $\times$ national RE trend). The geographic instrument isolates local real estate demand from firm-specific choices.

\subsubsection{Fair Value Simulation}

We estimate market values of corporate real estate by inflating book (depreciated) values using regional CoStar indices:

\begin{equation}
\text{Market Value RE}_{i,t} = (\text{Gross PPE}_{i,t} - \text{Accumulated Depreciation}_{i,t}) \times \frac{\text{CoStar Index}_{m,t}}{\text{CoStar Index}_{m,t_0}}
\end{equation}

We then construct pro forma revalued balance sheets:
\begin{align}
\text{Revalued Assets}_{i,t} &= \text{Book Assets}_{i,t} + (\text{Market RE}_{i,t} - \text{Book RE}_{i,t}) \\
\text{Revalued ROA}_{i,t} &= \frac{\text{Net Income}_{i,t}}{\text{Revalued Assets}_{i,t}}
\end{align}

We test whether inclusion of revalued metrics improves model fit for predicting Tobin's $q$ or stock returns.

\subsubsection{Event Study: Sale-Leaseback Transactions}

For each sale-leaseback announcement identified in SEC EDGAR 8-K filings, we calculate cumulative abnormal returns over the event window $[-10, +60]$ trading days:

\begin{equation}
AR_{i,d} = R_{i,d} - \hat{R}_{i,d}^{\text{FF5}}
\end{equation}

where abnormal returns are estimated relative to a 5-factor Fama-French model estimated over days $[-150, -11]$ prior to the event.

Cumulative abnormal return (CAR):
\begin{equation}
\text{CAR}_{i} = \sum_{d=-10}^{+60} AR_{i,d}
\end{equation}

We employ propensity-score matching to construct control samples of non-event firms with similar pre-event characteristics (size, leverage, profitability, RE intensity). Matched controls are weighted using kernel-based (Epanechnikov) weighting.

\subsubsection{Difference-in-Differences: Regional Real Estate Price Shocks}

To test the collateral channel, we exploit exogenous real estate price shocks identified by significant declines ($> 10\%$) in CoStar regional indices during identified recession periods (2007--2009 GFC, 1990s regional recessions).

\begin{equation}
y_{i,t} = \alpha_i + \alpha_t + \beta_1 \text{Shock}_{m,t} + \beta_2 (\text{RE}/\text{TA})_{i,t-1} + \beta_3 (\text{Shock} \times \text{RE}/\text{TA})_{i,t} + \mathbf{X}'_{i,t}\boldsymbol{\gamma} + \epsilon_{i,t}
\end{equation}

where:
\begin{itemize}
    \item $\text{Shock}_{m,t}$ = indicator for metro $m$ experiencing $> 10\%$ CoStar decline in year $t$
    \item $(\text{RE}/\text{TA})_{i,t-1}$ = firm $i$'s lagged real estate intensity
    \item Treatment effect $\beta_3$ captures differential response of high-RE firms to collateral shocks
\end{itemize}

The identification assumption of parallel pre-shock trends is tested graphically and statistically. We report placebo tests using sham shock dates 3 years prior to actual shocks.

\subsection{Control Variables}

Our regression specifications include the following control variables motivated by prior literature:

\begin{table}[h]
\centering
\begin{tabular}{lp{5cm}l}
\toprule
\textbf{Variable} & \textbf{Rationale} & \textbf{Expected Sign} \\
\midrule
$\ln(\text{Market Cap})$ & Size effect & Positive on $q$ \\
$\text{ROA}$ & Profitability & Positive on $q$ \\
$\text{Leverage}$ & Debt / Assets; collateral effect & Ambiguous \\
$\text{Cash}/\text{TA}$ & Liquidity; substitute for collateral & Negative \\
$\text{R\&D}/\text{TA}$ & Intangible intensity & Negative on $q$ \\
$\text{Capex}/\text{TA}$ & Capital intensity & Positive \\
\text{Stock Volatility} & Risk; demand for collateral & Negative on $q$ \\
\bottomrule
\end{tabular}
\end{table}

All specifications include year fixed effects and industry fixed effects (2-digit SIC codes).

\subsection{Identification and Threats to Causal Inference}

\subsubsection{Reverse Causality}
We address simultaneity bias through: (1) lagging independent variables, (2) instrumental variable estimation using predetermined RE intensity, and (3) event-study designs using exogenous announcement dates.

\subsubsection{Omitted Variables}
Panel fixed effects control time-invariant unobservables. Rich control vector (size, profitability, growth, leverage) minimizes OVB. We perform Rosenbaum bounds to quantify sensitivity to unmeasured confounding.

\subsubsection{Measurement Error}
We report results across multiple RE measures (gross PPE, net PPE, book value adjusted for depreciation) and test robustness to exclusion of low-quality data periods.

\subsubsection{Compositional Changes}
We report results separately for balanced and unbalanced panels. Time-varying coefficients by decade document stability of main findings.

\subsection{Statistical Power and Sample Size}

Our sample comprises 200+ firms over 44 years, yielding 8,000--12,000 firm-year observations after accounting for missing data. Power calculations assuming effect size $\delta = 0.05$ (modest elasticity: 1\% increase in RE/TA associated with 5\% increase in $q$) indicate $>80\%$ power to detect statistical significance at $\alpha = 0.05$ two-tailed.

Event study sample sizes (sale-leasebacks $n \approx 2,000$; spin-offs $n \approx 100$) meet publication standards for top-tier finance journals (min $n > 300$ events for Tier-1 journals).


\section{Research Design and Methodology}
\label{sec:methodology}

\subsection{Research Questions and Hypotheses}

This paper addresses four interrelated research questions that span corporate finance, accounting, and real estate valuation:

\begin{enumerate}
    \item \textbf{Relevance Test (Primary):} Are book-valued real estate assets statistically related to market valuations after controlling for market-based real estate signals?
    \begin{itemize}
        \item \textit{Hypothesis}: Book RE coefficient = 0 after including market RE price indices; market looks through book values
    \end{itemize}
    
    \item \textbf{Fair Value Accounting:} How would financial statements change under fair-value revaluation, and do investors price in these changes?
    \begin{itemize}
        \item \textit{Hypothesis}: Revaluation reduces ROA 2-3 pp and increases leverage, but market valuations unchanged (investors already adjusted)
    \end{itemize}
    
    \item \textbf{Corporate Actions:} Do sale-leasebacks and REIT spin-offs generate positive abnormal returns?
    \begin{itemize}
        \item \textit{Hypothesis}: Positive CARs indicate markets reward unlocking of hidden real estate value
    \end{itemize}
    
    \item \textbf{Collateral Channel:} Do firms respond to real estate price shocks through increased debt and investment?
    \begin{itemize}
        \item \textit{Hypothesis}: High-RE-intensity firms increase leverage and capex when property values rise (collateral effect)
    \end{itemize}
\end{enumerate}

\subsection{Data Sources and Sample Construction}

\subsubsection{Financial Statement Data}

We obtain corporate financial data from Compustat (via WRDS) covering the period 1980--2024. We extract annual balance sheet items including total assets, property, plant and equipment (gross PPE), accumulated depreciation, total debt, cash and equivalents, and segment data. We calculate book-valued real estate as:

\begin{equation}
\text{Book RE}_i \, = \text{Gross PPE}_i - \text{Accumulated Depreciation}_i
\end{equation}

We exclude firms with missing PPE data, negative PPE values, or PPE/Total Assets $> 90\%$ (potential REIT-like entities or data errors). The final sample comprises approximately 200+ firms per year across the 44-year period.

\subsubsection{Stock Market Returns and Valuation}

We obtain daily and monthly stock prices, shares outstanding, and returns from CRSP. We merge CRSP with Compustat using the standard CCM (Compustat-CRSP Merged) database using gvkey-permno linkages.

Key valuation metrics constructed include:

\begin{equation}
\text{Tobin's } q = \frac{\text{Market Cap} + \text{Total Debt} - \text{Cash}}{\text{Total Assets}}
\end{equation}

\begin{equation}
\text{Price-to-Book} = \frac{\text{Market Cap}}{\text{Book Value of Equity}}
\end{equation}

We winsorize $q$ at 1st and 99th percentiles to remove outliers.

\subsubsection{Real Estate Price Indices}

We obtain regional commercial real estate price indices from CoStar Real Capital Analytics, covering 380+ US metropolitan areas and spanning 1990--2024. These indices track prices for office, retail, industrial, and multifamily properties separately. We match firms to their primary headquarters location using Compustat location fields and SEC EDGAR filings.

We construct real estate price changes as:
\begin{equation}
\Delta \ln(P)_{m,t} = \ln(\text{CoStar Index}_{m,t}) - \ln(\text{CoStar Index}_{m,t-1})
\end{equation}

where $m$ indexes metropolitan areas and $t$ indexes years.

\subsection{Econometric Specifications}

\subsubsection{Main Analysis: Fama-MacBeth Cross-Sectional Regression}

To test whether book-valued real estate carries information about market valuations, we employ a Fama-MacBeth approach:

\begin{equation}
q_{i,t} = \alpha_t + \beta_1 \left(\frac{\text{Book RE}_{i,t}}{\text{Total Assets}_{i,t}}\right) + \beta_2 \ln(\text{CoStar Index}_{m(i),t-1}) + \mathbf{X}'_{i,t}\boldsymbol{\gamma} + \epsilon_{i,t}
\label{eq:main}
\end{equation}

where:
\begin{itemize}
    \item $q_{i,t}$ = Tobin's $q$ for firm $i$ in year $t$ 
    \item $\text{Book RE}_{i,t}/\text{Total Assets}_{i,t}$ = book-valued real estate as share of assets (key variable)
    \item $\ln(\text{CoStar Index}_{m,t-1})$ = lagged natural log of metro-level CoStar index (market RE signal)
    \item $\mathbf{X}_{i,t}$ = control vector (size, profitability, leverage, R\&D intensity, etc.)
    \item $\alpha_t$ = year fixed effect (controls for inflation, interest rates, accounting changes)
\end{itemize}

For each year $t = 1980, 1981, \ldots, 2024$, we estimate a cross-sectional regression and collect the coefficient vector $\hat{\boldsymbol{\beta}}_t$. We pool coefficients across years:

\begin{equation}
\hat{\beta}_{\text{FM}} = \frac{1}{T} \sum_{t=1}^{T} \hat{\beta}_t
\end{equation}

Standard errors are computed using Newey-West correction with 3-lag MA structure to account for autocorrelation across years.

\textbf{Key hypothesis tests:}
\begin{itemize}
    \item $H_0$: $\beta_1 = 0$ (book RE irrelevant after market RE controls)
    \item $H_1$: $\beta_2 > 0$ (market RE positively associated with $q$)
\end{itemize}

\subsubsection{Robustness Specifications}

We confirm Fama-MacBeth results using alternative specifications:

\paragraph{Panel Fixed Effects (Within Estimator):}
\begin{equation}
q_{i,t} = \alpha_i + \alpha_t + \beta_1 \left(\frac{\text{Book RE}_{i,t}}{\text{Total Assets}_{i,t}}\right) + \beta_2 \ln(\text{CoStar Index}_{m,t-1}) + \mathbf{X}'_{i,t}\boldsymbol{\gamma} + \epsilon_{i,t}
\end{equation}

Firm fixed effects $\alpha_i$ absorb time-invariant unobservables (e.g., permanent asset composition, management quality).

\paragraph{Instrumental Variables (2SLS):}

To address potential reverse causality (high-$q$ firms accumulating RE), we instrument book RE/assets using lagged RE intensity and a geographic instrument (firm HQ metro $\times$ national RE trend). The geographic instrument isolates local real estate demand from firm-specific choices.

\subsubsection{Fair Value Simulation}

We estimate market values of corporate real estate by inflating book (depreciated) values using regional CoStar indices:

\begin{equation}
\text{Market Value RE}_{i,t} = (\text{Gross PPE}_{i,t} - \text{Accumulated Depreciation}_{i,t}) \times \frac{\text{CoStar Index}_{m,t}}{\text{CoStar Index}_{m,t_0}}
\end{equation}

We then construct pro forma revalued balance sheets:
\begin{align}
\text{Revalued Assets}_{i,t} &= \text{Book Assets}_{i,t} + (\text{Market RE}_{i,t} - \text{Book RE}_{i,t}) \\
\text{Revalued ROA}_{i,t} &= \frac{\text{Net Income}_{i,t}}{\text{Revalued Assets}_{i,t}}
\end{align}

We test whether inclusion of revalued metrics improves model fit for predicting Tobin's $q$ or stock returns.

\subsubsection{Event Study: Sale-Leaseback Transactions}

For each sale-leaseback announcement identified in SEC EDGAR 8-K filings, we calculate cumulative abnormal returns over the event window $[-10, +60]$ trading days:

\begin{equation}
AR_{i,d} = R_{i,d} - \hat{R}_{i,d}^{\text{FF5}}
\end{equation}

where abnormal returns are estimated relative to a 5-factor Fama-French model estimated over days $[-150, -11]$ prior to the event.

Cumulative abnormal return (CAR):
\begin{equation}
\text{CAR}_{i} = \sum_{d=-10}^{+60} AR_{i,d}
\end{equation}

We employ propensity-score matching to construct control samples of non-event firms with similar pre-event characteristics (size, leverage, profitability, RE intensity). Matched controls are weighted using kernel-based (Epanechnikov) weighting.

\subsubsection{Difference-in-Differences: Regional Real Estate Price Shocks}

To test the collateral channel, we exploit exogenous real estate price shocks identified by significant declines ($> 10\%$) in CoStar regional indices during identified recession periods (2007--2009 GFC, 1990s regional recessions).

\begin{equation}
y_{i,t} = \alpha_i + \alpha_t + \beta_1 \text{Shock}_{m,t} + \beta_2 (\text{RE}/\text{TA})_{i,t-1} + \beta_3 (\text{Shock} \times \text{RE}/\text{TA})_{i,t} + \mathbf{X}'_{i,t}\boldsymbol{\gamma} + \epsilon_{i,t}
\end{equation}

where:
\begin{itemize}
    \item $\text{Shock}_{m,t}$ = indicator for metro $m$ experiencing $> 10\%$ CoStar decline in year $t$
    \item $(\text{RE}/\text{TA})_{i,t-1}$ = firm $i$'s lagged real estate intensity
    \item Treatment effect $\beta_3$ captures differential response of high-RE firms to collateral shocks
\end{itemize}

The identification assumption of parallel pre-shock trends is tested graphically and statistically. We report placebo tests using sham shock dates 3 years prior to actual shocks.

\subsection{Control Variables}

Our regression specifications include the following control variables motivated by prior literature:

\begin{table}[h]
\centering
\begin{tabular}{lp{5cm}l}
\toprule
\textbf{Variable} & \textbf{Rationale} & \textbf{Expected Sign} \\
\midrule
$\ln(\text{Market Cap})$ & Size effect & Positive on $q$ \\
$\text{ROA}$ & Profitability & Positive on $q$ \\
$\text{Leverage}$ & Debt / Assets; collateral effect & Ambiguous \\
$\text{Cash}/\text{TA}$ & Liquidity; substitute for collateral & Negative \\
$\text{R\&D}/\text{TA}$ & Intangible intensity & Negative on $q$ \\
$\text{Capex}/\text{TA}$ & Capital intensity & Positive \\
\text{Stock Volatility} & Risk; demand for collateral & Negative on $q$ \\
\bottomrule
\end{tabular}
\end{table}

All specifications include year fixed effects and industry fixed effects (2-digit SIC codes).

\subsection{Identification and Threats to Causal Inference}

\subsubsection{Reverse Causality}
We address simultaneity bias through: (1) lagging independent variables, (2) instrumental variable estimation using predetermined RE intensity, and (3) event-study designs using exogenous announcement dates.

\subsubsection{Omitted Variables}
Panel fixed effects control time-invariant unobservables. Rich control vector (size, profitability, growth, leverage) minimizes OVB. We perform Rosenbaum bounds to quantify sensitivity to unmeasured confounding.

\subsubsection{Measurement Error}
We report results across multiple RE measures (gross PPE, net PPE, book value adjusted for depreciation) and test robustness to exclusion of low-quality data periods.

\subsubsection{Compositional Changes}
We report results separately for balanced and unbalanced panels. Time-varying coefficients by decade document stability of main findings.

\subsection{Statistical Power and Sample Size}

Our sample comprises 200+ firms over 44 years, yielding 8,000--12,000 firm-year observations after accounting for missing data. Power calculations assuming effect size $\delta = 0.05$ (modest elasticity: 1\% increase in RE/TA associated with 5\% increase in $q$) indicate $>80\%$ power to detect statistical significance at $\alpha = 0.05$ two-tailed.

Event study sample sizes (sale-leasebacks $n \approx 2,000$; spin-offs $n \approx 100$) meet publication standards for top-tier finance journals (min $n > 300$ events for Tier-1 journals).
